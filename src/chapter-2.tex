\documentclass{article}
\usepackage[utf8]{inputenc}

\usepackage{amssymb}
\usepackage{amsmath}

% for bold italic matrices and vectors
\usepackage{bm}
\def\ma{{\bm{A}}}
\def\mb{{\bm{B}}}
\def\mc{{\bm{C}}}
\def\mx{{\bm{X}}}

\def\R{{\mathbb{R}}}
\def\I{{\mathbb{I}}}

\newcommand{\matr}[1]{\bm{#1}}

% shamelessly stolen from https://tex.stackexchange.com/questions/2233/whats-the-best-way-make-an-augmented-coefficient-matrix
\makeatletter
\renewcommand*\env@matrix[1][*\c@MaxMatrixCols c]{%
  \hskip -\arraycolsep
  \let\@ifnextchar\new@ifnextchar
  \array{#1}}
\makeatother


\newcommand\undermat[2]{%
  \makebox[0pt][l]{$\smash{\underbrace{\phantom{%
    \begin{matrix}#2\end{matrix}}}_{\text{$#1$}}}$}#2}


\title{\LARGE Mathematics for Machine Learning - Solutions\\
    \Large Chapter 2 - Linear Algebra}
\author{João Tomás Caldeira}
\date{}

\begin{document}

\maketitle

\textbf{2.1.}

\textbf{a)} We first attempt to show the group properties:

\begin{enumerate}
    \item Closure
    
    For the set $\R \setminus \{-1\}$ to be closed under $\star$, the latter should take two of its elements and yield a number different from -1.
    
    Let us assume that we can get -1 from applying the operator to some $a, b \in \R \setminus \{-1\}$. Then,
    \begin{equation}
        ab + a + b = -1 \iff ab + a + b + 1 = 0 \iff (a + 1)(b + 1) = 0,
    \end{equation}
    
    so either $a = -1$ or $b = -1$. We have reached a contradiction; recall that we introduced the assumption that $a \neq -1$ and $b \neq -1$. Therefore, $\R \setminus \{-1\}$ is closed under $\star$.
    
    \item Associativity
    
    \begin{equation}
    \begin{split}
        (a \star b) \star c & = (ab + a + b) \star c \\ & =  (ab + a + b)c + (ab + a + b) + c \\ & = a + ab + abc + ac + b + bc + c \\ & = (abc + ab + ac) + (bc + b + c) + a \\ & = a(b \star c) + a + (b \star c) \\ & = a \star (b \star c)
    \end{split}
    \end{equation}
    
    \item Neutral element
    
    $\forall a \in \R \setminus \{-1\}: a \star 0 = a \cdot 0 + a + 0 = a$, and $0 \star a = 0 \cdot a + 0 + a = a$, so 0 is the neutral element of $(\R \setminus \{-1\}, \star)$.
    
    \item Inverse element
    We try to find some $x$ for which $\forall a \in \R \setminus \{-1\}: a \star x = x \star a = 0$. We have
    \begin{equation}\label{2.1.a.inverse}
    \begin{split}
        a \star x = 0 \\ & \iff ax + a + x = 0 \\ & \iff x = -\frac{a}{a + 1}.
    \end{split}
    \end{equation}
    
    Since $a \neq -1$ necessarily, the solution for equation \ref{2.1.a.inverse} always exists. Thus, $x$ is the inverse element of $(\R \setminus \{-1\}, \star)$. \textit{This concludes proof that $(\R \setminus \{-1\}, \star)$ is a group.}
    
    \item Commutativity
    
    $\forall a, b \in \R \setminus \{-1\}: a \star b = ab + a + b = ba + b + a = b \star a.$
    
    Alternatively, we may observe that $a \star b$ = $b \star a$, by the commutativity of addition and multiplication of real numbers.
\end{enumerate}

We conclude that $(\R \setminus \{-1\}, \star)$ is an Abelian group.

\textbf{b)} \begin{equation}
    \begin{split}
        3 \star x \star x = 15 & \iff (3x + 3 + x) \star x = 15 \\
        & \iff 3x^2 + 3x + x^2 + 3x + 3 + x + x = 15 \\
        & \iff 4x^2 + 8x - 12 = 0 \\
        & \iff x = 1 \vee x = -3
    \end{split}
\end{equation}

Therefore, $x \in \{-3, 1\}$.

\textbf{2.2.} HELP.

\textbf{2.3.} Let $\ma, \mb$ be in $\mathcal{G}$ such that

\[
\ma = \begin{bmatrix}
    1 & x & z\\
    0 & 1 & y\\
    0 & 0 & 1
\end{bmatrix},\ 
\mb = \begin{bmatrix}
    1 & a & c\\
    0 & 1 & b\\
    0 & 0 & 1
\end{bmatrix},\text{ with } x, y, z, a, b, c \in \R
\]

Then,

\[
\ma \cdot \mb = \begin{bmatrix}
    1 & a + x & c + bx + z\\
    0 & 1 & b + y\\
    0 & 0 & 1
\end{bmatrix}
\]

\begin{enumerate}
    \item Closure
    
    Since $a + x, b + y, c + bx + z \in \R$, then $\ma \cdot \mb \in \mathcal{G}$. Therefore, $\mathcal{G}$ is closed under $\cdot$ (matrix multiplication).
    
    \item Associativity
    
    Let us ake $\ma, \mb$ as defined above, and let $\mc$ be in $\mathcal{G}$ such that
    
    \[
    \mc = \begin{bmatrix}
        1 & p & r\\
        0 & 1 & q\\
        0 & 0 & 1
    \end{bmatrix},\text{ with } p, q, r \in \R.
    \]
    
    Then,
    
    \begin{equation}
    \begin{split}
    (\ma \cdot \mb) \cdot \mc = & \begin{bmatrix}
        1 & a + x & c + bx + z\\
        0 & 1 & b + y\\
        0 & 0 & 1
    \end{bmatrix}\begin{bmatrix}
        1 & p & r\\
        0 & 1 & q\\
        0 & 0 & 1
    \end{bmatrix}\\ = & \begin{bmatrix}
        1 & p + a + x & r + qa + qx + c + bx + z\\
        0 & 1 & q + b + y\\
        0 & 0 & 1
    \end{bmatrix}.
    \end{split}
    \end{equation}
    
    Similarly, $\ma \cdot (\mb \cdot \mc)$ yields the same result. Therefore, $\cdot$ is associative.
    
    \item Neutral element
    
    $\I \cdot \ma = \ma = \ma \cdot \I, \forall \ma \in \mathcal{G}.$
    
    Therefore, $\I$ is the neutral element.
    
    \item Inverse element
    
    We need to find the inverse such that
    
    \[
    \ma \cdot \ma^{-1} = \I\text{ (neutral element), and } \ma^{-1} \cdot \ma = \I.
    \]
    
    To find the \textit{right} inverse $\ma_r^{-1}$ of $\ma$, which needs to satisfy $\ma \cdot \ma_r^{-1} = \I$, we solve the linear system $\ma \cdot \mx = \I$. We write the system in augmented notation $[\ma | \I]$ and solve using Gaussian Elimination (G.E.), which yields
    
    \[
    [\ma | \I] \overset{G.E.}{=} \begin{bmatrix}[c c c | c c c]
        1 & 0 & 0 & 1 & -x & xy - z\\
        0 & 1 & 0 & 0 & 1 & -y\\
        0 & 0 & 1 & 0 & 0 & 1
    \end{bmatrix}\text{, so } \ma_r^{-1} = \begin{bmatrix}
        1 & -x & xy - z\\
        0 & 1 & -y\\
        0 & 0 & 1
    \end{bmatrix} \in \mathcal{G}.
    \]
    
    Since the inverse element is unique, if there is a left inverse $\ma_l^{-1}$ (i.e., such that $\ma_l^{-1}\ma = \I$), then it is equal to the right inverse $\ma_r^{-1}$. However, since matrix multiplication is not commutative, we need to check that $\ma_r^{-1}\ma = \I$ indeed:
    
    \begin{equation}
    \begin{split}
    \ma_r^{-1}\ma & = \begin{bmatrix}
        1 & -x & xy - z\\
        0 & 1 & -y\\
        0 & 0 & 1
    \end{bmatrix} \begin{bmatrix}
        1 & x & z\\
        0 & 1 & y\\
        0 & 0 & 1
    \end{bmatrix}\\ & = \begin{bmatrix}
        1 & x - x & z - xy + xy - z\\
        0 & 1 & y - y\\
        0 & 0 & 1
    \end{bmatrix}\\ & = \begin{bmatrix}
        1 & 0 & 0\\
        0 & 1 & 0\\
        0 & 0 & 1
    \end{bmatrix} = \I.
    \end{split}
    \end{equation}
    
    Thus, every element of $\mathcal{G}$ has an inverse. \textit{This concludes proof that ($\mathcal{G}$, $\cdot$) is a group.}
    
    \item Commutativity
    
    We need only find some $\ma, \mb \in \mathcal{G}$ for which $\ma \cdot \mb \neq \mb \cdot \ma$ to show that $\cdot$ is not commutative. Using $\ma, \mb$ as defined above, we have
    
    \[
    \mb \cdot \ma = \begin{bmatrix}
        1 & x + a & z + ax + c\\
        0 & 1 & y + b\\
        0 & 0 & 1
    \end{bmatrix},
    \]
    
    which differs from $\ma \cdot \mb$ in the element in the position $1, 3$. In particular, e.g., for $A = \begin{bmatrix}
        1 & 3 & 0\\
        0 & 1 & 0\\
        0 & 0 & 1
    \end{bmatrix}, B = \begin{bmatrix}
        1 & 1 & 0\\
        0 & 1 & 0\\
        0 & 0 & 1
    \end{bmatrix}$, we have
    
    \[
    \ma \cdot \mb = \begin{bmatrix}
        1 & 4 & 0\\
        0 & 1 & 0\\
        0 & 0 & 1
    \end{bmatrix} \neq \mb \cdot \ma = \begin{bmatrix}
        1 & 4 & 3\\
        0 & 1 & 0\\
        0 & 0 & 1
    \end{bmatrix}
    \]
    
    Therefore, $\cdot$ is not commutative, and thus $(\mathcal{G}, \cdot)$ is a non-Abelian group.
\end{enumerate}

\textbf{2.4.}

\textbf{a)} The matrices have dimensions $(3 \times 2)$ and $(3 \times 3)$, respectively. Therefore, the matrix product is not defined for these two matrices.

\textbf{b)}

\[
\begin{bmatrix}
    1 & 2 & 3\\
    4 & 5 & 6\\
    7 & 8 & 9
\end{bmatrix}
\begin{bmatrix}
    1 & 1 & 0\\
    0 & 1 & 1\\
    1 & 0 & 1
\end{bmatrix} = \begin{bmatrix}
    4 & 3 & 5\\
    10 & 9 & 11\\
    16 & 15 & 17
\end{bmatrix}
\]

\textbf{c)}

\[
\begin{bmatrix}
    1 & 1 & 0\\
    0 & 1 & 1\\
    1 & 0 & 1
\end{bmatrix}
\begin{bmatrix}
    1 & 2 & 3\\
    4 & 5 & 6\\
    7 & 8 & 9
\end{bmatrix} = \begin{bmatrix}
    5 & 7 & 9\\
    11 & 13 & 15\\
    8 & 10 & 12
\end{bmatrix}
\]

\textbf{d)}

\[
\begin{bmatrix}
    1 & 2 & 1 & 2\\
    4 & 1 & -1 & -4
\end{bmatrix}
\begin{bmatrix}
    0 & 2\\
    1 & -1\\
    2 & 1\\
    5 & 2
\end{bmatrix} = \begin{bmatrix}
    14 & 6\\
    -21 & 2
\end{bmatrix}
\]

\textbf{e)}

\[
\begin{bmatrix}
    0 & 3\\
    1 & -1\\
    2 & 1\\
    5 & 2
\end{bmatrix}
\begin{bmatrix}
    1 & 2 & 1 & 2\\
    4 & 1 & -1 & 4
\end{bmatrix} = \begin{bmatrix}
    12 & 3 & -3 & -12\\
    -3 & 1 & 2 & 6\\
    6 & 5 & 1 & 0\\
    13 & 12 & 3 & 2
\end{bmatrix}
\]

\textbf{2.5.} For each of the systems of linear equations, we start by writing them in augmented matrix notation and perform Gaussian Elimination to obtain reduced row-echelon form.

\textbf{a)}

\[
\begin{bmatrix}[c c c c | c]
    1 & 1 & -1 & -1 & 1\\
    2 & 5 & -7 & -5 & -2\\
    2 & -1 & 1 & 3 & 4\\
    5 & 2 & -4 & 2 & 6
\end{bmatrix} \rightarrow \begin{bmatrix}[c c c c | c]
    1 & 1 & -1 & -1 & 1\\
    0 & 3 & -5 & -3 & -4\\
    0 & 0 & -2 & 2 & -2\\
    0 & 0 & 0 & 0 & 1
\end{bmatrix}
\]

We can see from the last row that the equation system is inconsistent and thus has no solution. Therefore, $S = \emptyset$.

\textbf{b)}

\[
\begin{bmatrix}[c c c c c | c]
    1 & -1 & 0 & 0 & 1 & 3\\
    1 & 1 & 0 & -3 & 0 & 6\\
    2 & -1 & 0 & 1 & -1 & 5\\
    -1 & 2 & 0 & -2 & -1 & -1
\end{bmatrix} \rightarrow \begin{bmatrix}[c c c c c | c]
    1 & 0 & 0 & 0 & -1 & 3\\
    0 & 0 & 0 & 1 & -1 & -1\\
    0 & 0 & 0 & 0 & 0 & 0\\
    0 & 1 & 0 & 0 & -2 & 0
\end{bmatrix}
\]

The third row is a tautology, and therefore redundant in the system. We remove it, rearrange the rows, and apply the "minus 1 trick":

\[
\begin{bmatrix}[c c c c c | c]
    1 & 0 & 0 & 0 & -1 & 3\\
    0 & 1 & 0 & 0 & -2 & 0\\
    0 & 0 & -1 & 0 & 0 & 0\\
    0 & 0 & 0 & 1 & -1 & -1\\
    0 & 0 & 0 & 0 & -1 & 0
\end{bmatrix}
\]

The right-hand side of the system is a particular solution, while the two columns with -1 on the diagonal show the directions of the solution space. The solution set is thus

\[
S = \left\{\matr{x} \in \R^5: \matr{x} = \begin{bmatrix}
    3\\0\\0\\-1\\0
\end{bmatrix} + \lambda_1\begin{bmatrix}
    0\\0\\-1\\0\\0
\end{bmatrix} + \lambda_2\begin{bmatrix}
    3\\0\\0\\-1\\0
\end{bmatrix}, \lambda_1, \lambda_2 \in \R\right\}
\]

\textbf{2.6.}

We start by writing the system of linear equations in augmented matrix form and apply Gaussian Elimination to obtain reduced row-echelon form:

\[
\begin{bmatrix}[c c c c c c | c]
    0 & 1 & 0 & 0 & 1 & 0 & 2\\
    0 & 0 & 0 & 1 & 1 & 0 & -1\\
    0 & 1 & 0 & 0 & 0 & 1 & 1
\end{bmatrix} \rightarrow
\begin{bmatrix}[c c c c c c | c]
    0 & 1 & 0 & 0 & 0 & 1 & 1\\
    0 & 0 & 0 & 1 & 0 & 1 & -2\\
    0 & 0 & 0 & 0 & 1 & -1 & 1
\end{bmatrix}
\]

Applying the "minus 1 trick", we get

\[
\begin{bmatrix}[c c c c c c | c]
    -1 & 0 & 0 & 0 & 0 & 0 & 0\\
    0 & 1 & 0 & 0 & 0 & 1 & 1\\
    0 & 0 & -1 & 0 & 0 & 0 & 0\\
    0 & 0 & 0 & 1 & 0 & 1 & -2\\
    0 & 0 & 0 & 0 & 1 & -1 & 1\\
    0 & 0 & 0 & 0 & 0 & -1 & 0
\end{bmatrix}
\]

The right-hand side of the system is a particular solution, while the three columns with -1 on the diagonal show the directions of - and thys span - the solution space. The solution set is then

\[
S = \left\{\matr{x} \in \R^6: \matr{x} = \begin{bmatrix}
    0\\1\\0\\-2\\1\\0
\end{bmatrix} + \lambda_1\begin{bmatrix}
    -1\\0\\0\\0\\0\\0
\end{bmatrix} + \lambda_2\begin{bmatrix}
    0\\0\\-1\\0\\0\\0
\end{bmatrix} + \lambda_3\begin{bmatrix}
    0\\1\\0\\1\\-1\\-1
\end{bmatrix}, \lambda_1, \lambda_2, \lambda_3 \in \R\right\}.
\]

\textbf{2.7.} We have $\ma\matr{x} = 12\matr{x} \iff \ma\matr{x} - 12\matr{x} = \matr{0} \iff (\ma - 12\I)\matr{x} = \matr{0}$. Writing this system of linear equations in augmented form, we have

\[
\begin{bmatrix}[c c c | c]
    -6 & 4 & 3 & 0\\
    6 & -12 & 9 & 0\\
    0 & 8 & -12 & 0
\end{bmatrix}
\]

We can also write the constraint $\sum_{i = 1}^3 x_i = 1 \iff x_1 + x_2 + x_3 = 1$ as another row (equation) in the above matrix, which we also solve to reduced row-echelon form:

\[
\begin{bmatrix}[c c c | c]
    -6 & 4 & 3 & 0\\
    6 & -12 & 9 & 0\\
    0 & 8 & -12 & 0\\
    1 & 1 & 1 & 1
\end{bmatrix} \rightarrow
\begin{bmatrix}[c c c | c]
    1 & 0 & 0 & 3\\
    0 & 1 & 0 & 3\\
    0 & 0 & 1 & 2
\end{bmatrix}
\]

Therefore, we have a unique solution $\begin{bmatrix}
    3 & 3 & 2
\end{bmatrix}^\top$.

\textbf{2.8.} IN PROGRESS...

\end{document}
